\section{Motivation}

In a prototype implementation or simulation module, we may need to 
perform comparison of individual steps, to suggest which
alternative implementation has lower CPU overhead than others.
When a fine measurement of number of CPU cycles~\cite{iodedup} is needed 
for a piece of code, or when we need to compare the number of CPU cycles
that two sets of instructions might take, we need a
method to measure the CPU cycles expended by the 
system-under-test for processing, in each case. For such cases, the timestamp
counter (TSC) can be accessed before and after the target set of instructions,
and the difference between the two acccessed values is the number of CPU 
cycles expended.

The timestamp counter is a 64-bit model-specific register (MSR) present in Intel
machines, beginning with the Pentium processor\cite{rdtscpm1}. It is incremented
on every clock tick, and is set to 0 upon machine reset. It can be 
accessed using the RDTSC (read time stamp counter)
instruction in Microsoft VC++ environments. However, in Linux environment, the
GCC compiler has no in-built RDTSC instruction, because of which we need
to write a piece of assembly code to directly access the 64-bit value. 
The value of the time stamp counter needs to be loaded into the EDX and
EAX registers (high-order 32 bits into EDX, low-order into EAX) 
and then read up from them into a local variable. Since the Linux GCC compiler
does not have the RDTSC instruction, its asm opcode (0x31) has to be used
so as to allow the processor to understand the instruction even if the 
compiler does not.

In this document, we present collated information regarding how-to use the
model-specific TSC register in Linux environment, to count the number
of ticks or CPU cycles spent for any target set of instructions. 
Our contributions in this document are the following:-
\begin{itemize}
\item Collation of information regarding CPU cycle measurement
\item Provision of re-usable code-snippets for quick CPU cycle measurement for any target set of instructions
\item Example usecase: Measurement of CPU cycles incurred for MD5 hashing.
\end{itemize}

In the next section, we present a brief survey of existing profiling tools
for Linux, and delineate how they fail to meet the needs of monitoring CPU
cycles. The rest of the document is organized as follows.
Section~\ref{our-approach} demonstrates how to access the time stamp counter
in C code, and presents code snippets that can be quickly re-used. In
Section~\ref{sample-usecase}, we present a sample usecase wherein, the
time stamp counter is accessed to measure the number of CPU cycles 
incurred for MD5 hashing of given content. Section~\ref{conclusions} concludes.
